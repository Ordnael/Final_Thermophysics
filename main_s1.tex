The heat diffusion process, by which the distribution of temperatures from an initial state evolves given a set of initial and boundary conditions, relies on knowing beforehand the parameters of the heat diffusion equation, whose values can be independent of the temperature or not according to the asusmptions of the model.\\

Two different directions can be taken to analyse heat diffusion processes: first, given all the inputs, the problem falls into calculating, or approximating, the behavior of the system along the time, with a fairly straightforward calculation methodology whose complexity depends on the approximation method being implicit or explicit; the second direction, so called an inverse problem, consists on determining the unknown inputs or factors that produce the observed results, and is this specific problem that acts as the base for multiple methodologies to find the temperature dependent behavior of the thermophysical properties.\\

The following report summarizes the fundamental concepts regarding the solution of the Invese Heat Conduction Problem, using as an example a Bycond test, where the thermal diffusivity is to be determined parting from a set of measured resultsand the proper setting of initial conditions; the results are obtained through a simulation built using Matlab, and the conclusions based on these are recopiled.